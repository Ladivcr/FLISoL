\documentclass[a4paper,10pt]{article}
\usepackage[utf8]{inputenc}

%opening
\title{FLISoL\\Introducción a la programación\\
(Con Python)}
\author{José Vidal Cardona Rosas\\
St Data Science\\
UNAM, ENES Morelia\\
vrosas832@gmail.com}

\begin{document}

\maketitle

\begin{abstract}
    Actualmente aprender a programar se está volviendo algo más obligatorio 
    que opcional, independientemente de si tu perfil es el de un Arquitecto, 
    un matemático o el de un estudiante de ciencias de la computación. Aprender 
    a programar es fundamental en un mundo rodeado por tecnología y en 
    constante cambio, resulta curioso que estemos rodeados de tecnología pero 
    no todos sabemos lo que hay detrás. Esa es la principal razón por la
    que he comenzado a redactar este pequeño documento.
    \\
    
    La razón por la que he elegido el lenguaje español para llevar a cabo 
    la redacción de este documento es porque hay muy pocos recursos disponibles
    en Internet para los hispanohablantes, así que ahora tienen uno más.
    \textbf{Cualquiera con conocimientos, básicos o avanzados, en el lenguaje 
    de programación Python puede aportar, es bienvenida cualquier mano extra.}
    \footnote{Podría contarles la historia del lenguaje Python
    y su creador, pero eso se encuentra lejos de los propósitos del presente 
    texto por lo que, si el lector quiere, lo puede buscar en Internet}
\end{abstract}

\section{Instalación de Python}

\end{document}
