\documentclass{beamer}
\usepackage[utf8]{inputenc}
% TEMA DE LA PRESENTACIÓN
\usetheme{split}
\useinnertheme{circles}
\usepackage{beamerthemeshadow}
% COLOR DE LA PÁGINA
\usestructuretemplate{\color{structure}}{}
\beamertemplateshadingbackground{yellow!50}{blue!50}
% DATOS DE PRESENTACIÓN
\title{El software libre en la vida diaría}
\subtitle{FLISoL 2020} 
\author{José Vidal Cardona Rosas}
\institute{ENES, Morelia Mich.} 
\begin{document}
    % Frame DATOS DE PRESENTACIÓN
    \begin{frame}
        \titlepage
    \end{frame}

    % Frame ESQUEMA DE LA PLÁTICA
    \section{Esquema de la plática}
    \begin{frame}
        \frametitle{Esquema de la plática}
        %\tableofcontents[currentsection]
        \begin{block}{Temas a conocer}
            \begin{itemize}
                \item \textbf{¿Qué es el Software Libre?}
                \begin{itemize}
                    \item \textbf{¿Y el Open Source?}
                \end{itemize}
                \item \textbf{Diferencias entre Software Libre 
                y Open Source}
                \begin{itemize}
                    \item \textbf{¿Qué objetivos persiguen?}
                \end{itemize}
                \item \textbf{Software Libre de uso diario}
                \item \textbf{¿Cómo me beneficia el uso de
                Software Libre?}
                \item \textbf{Limitantes del Software Libre
                frente al Software de paga}
                \item \textbf{Dando el salto: Pasando de ser
                consumidor a creador}
            \end{itemize}
        \end{block}
    \end{frame}

    % Frame DEFINICIONES
    %\section{Definiciones}
    \begin{frame}
        \frametitle{Algunas definiciones antes de empezar}
    
            \textbf{Software:} Dicese de aquello que al fallar
            hace que golpees al hardware
       
            \textbf{Sistema Operativo/SO:} Es el software principal que gestiona 
            los recursos de hardware y nos ofrece servicios de aplicación. 
      
            \textbf{GNU/Linux:}  Un conjunto de SO libres y multiplataforma.
       
            \textbf{GNU:} SO de tipo Unix, con una gran colección de 
            programas informáticos que componen al sistema.
        
            %\textbf{Distro/Distribución:} Otra versión del SO basada en el núcleo de 
            %Linux con software precargado para satisfacer las necesidades del usuario. 
      
    \end{frame}
    \section{Software Libre Y Open Source}
    \begin{frame}
        \frametitle{¿Qué es el Software libre?}
        \framesubtitle{¿Y el Open Source?}
        
    \end{frame}


    % Botón de play para vídeo
    %\beamergotobutton{Vídeo}

\end{document}
